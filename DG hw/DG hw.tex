\documentclass[11pt]{article}
\usepackage[margin=1in]{geometry}
\usepackage{amsmath,amssymb,amsthm,mathtools}
\usepackage{enumitem}
\usepackage{hyperref}

\newtheorem{claim}{Claim}
\newtheorem{lemma}{Lemma}
\newtheorem{prop}{Proposition}
\newtheorem{thm}{Theorem}
\theoremstyle{remark}
\newtheorem*{remark}{Remark}

\newcommand{\ip}[2]{\left\langle #1,\,#2\right\rangle}
\newcommand{\norm}[1]{\left\lVert #1\right\rVert}
\newcommand{\Exp}{\mathrm{exp}}
\newcommand{\expp}{\mathrm{exp}_p}
\newcommand{\expy}{\mathrm{exp}_y}
\newcommand{\expphi}{\mathrm{exp}_{\phi(p)}}
\newcommand{\dist}{d_g}
\newcommand{\distt}{d_{\tilde g}}
\newcommand{\R}{\mathbb{R}}

\title{Homework 8 \vspace{-0.35em}}
\author{Jiasong Zhu}
\date{\today}


\begin{document}
\maketitle

\section*{Problem 1}
\begin{proof}[Solution]
    \textbf{(i)} 
Use coordinates $u=(u^1,\dots,u^n)$ on $U$ and the parametrization
\[
X(u) = (u^1,\dots,u^n,f(u)).
\]
Then the coordinate tangent vectors are
\[
X_i := \frac{\partial X}{\partial u^i}
     = e_i + f_i e_{n+1}, 
\qquad 
f_i := \frac{\partial f}{\partial u^i}.
\]

The induced metric is
\[
g_{ij} = \langle X_i,X_j\rangle
       = \delta_{ij} + f_i f_j.
\]

A normal vector is $(-f_1,\dots,-f_n,1)$, whose length is
\[
W := \sqrt{1+\sum_{k=1}^n f_k^2}.
\]
So the upward unit normal is
\[
N = \frac{1}{W}(-f_1,\dots,-f_n,1).
\]

The second fundamental form is, by Lee, Prop.~8.23
\[
h(X_i,X_j)
 = \biggl\langle \frac{\partial^2 X}{\partial u^i\partial u^j}, N\biggr\rangle.
\]
We have
\[
\frac{\partial^2 X}{\partial u^i\partial u^j}
 = f_{ij} e_{n+1},
\qquad 
f_{ij} := \frac{\partial^2 f}{\partial u^i\partial u^j},
\]
so
\[
h_{ij} := h(X_i,X_j)
 = \left\langle f_{ij} e_{n+1}, N\right\rangle
 = f_{ij} N_{n+1}
 = \frac{f_{ij}}{W}.
\]

By def of the shape operator $s$,
\[
\langle sX_i, X_j\rangle = h(X_i,X_j),
\]
so if $s(X_j) = s^k{}_j X_k$, then
\[
h_{ij} = g_{ik} s^k{}_j.
\]
Thus, the components of the shape operator in graph coordinates are
\[
s^k{}_j = g^{ki} h_{ij}
        = \frac{1}{W}\,g^{ki} f_{ij},
\]
where $(g^{ij})$ is the inverse matrix of $(g_{ij})=(\delta_{ij}+f_i f_j)$.

\textbf{(ii)} Now let
\[
f(x) = |x|^2 = \sum_{i=1}^n (x^i)^2.
\]
Then
\[
f_i = \frac{\partial f}{\partial x^i} = 2x^i, 
\qquad
f_{ij} = \frac{\partial^2 f}{\partial x^i\partial x^j} = 2\delta_{ij}.
\]
Hence
\[
|\nabla f|^2 = \sum_i f_i^2 = 4|x|^2 = 4r^2,\quad r:=|x|,
\qquad
W = \sqrt{1+4r^2}.
\]

From (i),
\[
h_{ij} = \frac{f_{ij}}{W} = \frac{2}{W}\,\delta_{ij}.
\]

The metric in these coordinates is
\[
g_{ij} = \delta_{ij} + f_i f_j = \delta_{ij} + 4x^i x^j,
\]
which is of the matrix
\[
G = I + 4x x^T.
\]

To find principal curvatures, we should find the eigenvalues of $s$, which is of the matrix
\[
S = (s^i{}_j) = G^{-1} H, \quad H=(h_{ij})=\frac{2}{W}I.
\]
So $S = \dfrac{2}{W} G^{-1}$, and its eigenvalues are $\dfrac{2}{W}$ times the eigenvalues of $G^{-1}$.

Choose an orthonormal basis in $\mathbb{R}^n$ so that
\[
e_1 = \frac{x}{r},\qquad e_2,\dots,e_n \perp x.
\]
Then
\[
G e_1 = (1+4r^2)e_1,\qquad
G e_\alpha = e_\alpha \quad (\alpha=2,\dots,n).
\]
So the eigenvalues of $G$ are $1+4r^2$ and $1$.  
Thus the eigenvalues of $G^{-1}$ are
\[
\frac{1}{1+4r^2},\qquad 1.
\]

Therefore the principal curvatures at a point with $|x|=r>0$ are
\[
\kappa_{\mathrm{rad}}(r) 
 = \frac{2}{W}\cdot\frac{1}{1+4r^2}
 = \frac{2}{(1+4r^2)^{3/2}},
\]
in the direction of $\nabla f$, and
\[
\kappa_{\mathrm{tan}}(r) 
 = \frac{2}{W} 
 = \frac{2}{\sqrt{1+4r^2}}
\]
in each of the $n-1$ directions orthogonal to $\nabla f$.

At the vertex $x=0$, we have $r=0$, so $G(0)=I$, $h_{ij}(0)=2\delta_{ij}$, hence
\[
S(0)=2I,
\]
and all principal curvatures at $x=0$ are
\[
\kappa_1(0)=\cdots=\kappa_n(0)=2.
\]
\end{proof}

\section*{Problem 2}
\begin{proof}
\textbf{(i)} Compute first derivatives:
\[
X_t = (a'\cos\theta,\; a'\sin\theta,\; b'),\qquad
X_\theta = (-a\sin\theta,\; a\cos\theta,\; 0).
\]

\[
E = \langle X_t,X_t\rangle = (a')^2 + (b')^2 = 1,
\]
\[
F = \langle X_t,X_\theta\rangle = 0,
\]
\[
G = \langle X_\theta,X_\theta\rangle = a^2.
\]
So in the basis $\{X_t,X_\theta\}$,
\[
g = \begin{pmatrix}1 & 0\\[2pt] 0 & a^2\end{pmatrix},
\]
and the coordinate lines $t=\text{const}$ (meridians) and $\theta=\text{const}$ (parallels) are orthogonal.

Compute
\[
X_t \times X_\theta = (-ab'\cos\theta,\; -ab'\sin\theta,\; aa'),
\]
whose length is $|X_t\times X_\theta| = a$. Thus a unit normal is
\[
N = \frac{1}{a}(X_t\times X_\theta)
  = (-b'\cos\theta,\; -b'\sin\theta,\; a').
\]
Second derivatives:
\[
X_{tt} = (a''\cos\theta,\; a''\sin\theta,\; b''),\qquad
X_{t\theta} = (-a'\sin\theta,\; a'\cos\theta,\; 0),
\]
\[
X_{\theta\theta} = (-a\cos\theta,\; -a\sin\theta,\; 0).
\]
Then
\[
e := h(X_t,X_t) = \langle X_{tt},N\rangle
= -a''b' + a'b'',
\]
\[
f := h(X_t,X_\theta) = \langle X_{t\theta},N\rangle = 0,
\]
\[
g := h(X_\theta,X_\theta) = \langle X_{\theta\theta},N\rangle = a b'.
\]

So, the matrices of the first and second fundamental forms in $\{X_t,X_\theta\}$ are
\[
\text{I} = \begin{pmatrix}1 & 0\\ 0 & a^2\end{pmatrix},
\qquad
\text{II} = \begin{pmatrix}-a''b'+a'b'' & 0\\ 0 & a b'\end{pmatrix}.
\]
The shape operator $S$ satisfies $\text{II}(v,w) = \langle Sv,w\rangle$.
In matrix form, $S = \text{I}^{-1}\text{II}$, so
\[
S = \begin{pmatrix}1 & 0\\ 0 & a^{-2}\end{pmatrix}
    \begin{pmatrix}-a''b'+a'b'' & 0\\ 0 & a b'\end{pmatrix}
  = \begin{pmatrix}
      -a''b' + a'b'' & 0\\[3pt]
      0 & \dfrac{b'}{a}
    \end{pmatrix}.
\]
Thus,
\[
S(X_t) = k_1 X_t,\quad k_1(t) = -a''(t)b'(t) + a'(t)b''(t),
\]
\[
S(X_\theta) = k_2 X_\theta,\quad k_2(t) = \frac{b'(t)}{a(t)}.
\]
So, $X_t$ and $X_\theta$ are eigenvectors of $S$. Therefore, the direction tangent to the meridian is a principal direction;
the direction tangent to the latitude circle is a principal direction.

\textbf{(ii)} For a surface in $\mathbb{R}^3$, the Gaussian curvature is
\[
K = \det S = k_1k_2.
\]
From above,
\[
k_1k_2
= \bigl(-a''b' + a'b''\bigr)\cdot \frac{b'}{a}
= \frac{b'}{a}\,\bigl(a'b'' - b'a''\bigr).
\]
Now use the unit-speed condition, we get
\[
(a')^2 + (b')^2 = 1.
\]
Differentiate it, we have
\[
2a'a'' + 2b'b'' = 0
\;\Longrightarrow\;
a'a'' + b'b'' = 0
\;\Longrightarrow\;
b'b'' = -a'a''.
\]
Then
\[
(a'b'' - b'a'')\,b'
= a'b''b' - b'^2 a''
= a'(-a'a'') - b'^2 a''
= -a'^2 a'' - b'^2 a''
= -(a'^2 + b'^2)a''
= -a''.
\]
So
\[
K = k_1k_2
  = \frac{(a'b'' - b'a'')b'}{a}
  = \frac{-a''}{a}.
\]
Thus, the Gaussian curvature at $X(t,\theta)$ is
\[
K(t,\theta) = -\dfrac{a''(t)}{a(t)}.
\]

\end{proof}

\section*{Problem 3}
\begin{proof}
\textbf{(i)} Let a unit–speed curve likes in problem 2 such that 
\[
\gamma(t) = (a(t),b(t)),\qquad (a')^2+(b')^2=1,\ a(t)>0,
\]
revolved around the $z$–axis gives a surface of revolution
\[
X(t,\theta) = (a(t)\cos\theta,\;a(t)\sin\theta,\;b(t)).
\]
From Problem~2 we know
\[
k_1(t) = -\frac{a''(t)}{b'(t)},\qquad
k_2(t) = \frac{b'(t)}{a(t)},\qquad
K(t,\theta) = k_1(t)k_2(t) = -\frac{a''(t)}{a(t)}.
\]
To get $K\equiv1$, we need
\[
-\frac{a''(t)}{a(t)} = 1 \quad\Longleftrightarrow\quad a''(t)+a(t)=0.
\]
For example, let 
\[
a(t)=2\cos t,
\]
and restrict to the interval $I=(-\pi/6,\pi/6)$ so that $a(t)>0$.
Define $b$ so that $\gamma$ is unit speed as
\[
(a')^2 + (b')^2 =1
\ \Longrightarrow\
4\sin^2 t + (b')^2 = 1
\ \Longrightarrow\
b'(t) = \sqrt{1-4\sin^2 t},
\]
which is smooth and positive on $I$. Let
\[
b(t) = \int_0^t \sqrt{1-4\sin^2 s}\,ds.
\]
Then $\gamma(t)=(a(t),b(t))$ is unit speed, $a>0$, and the corresponding surface
$S$ is a surface of revolution.
For this $a$ we have $a''(t)=-2\cos t = -a(t)$, so
\[
K(t,\theta) = -\frac{a''(t)}{a(t)} = 1
\]
for all $(t,\theta)\in I\times(0,2\pi)$.
Now look at one principal curvature
\[
k_2(t) = \frac{b'(t)}{a(t)}
            = \frac{\sqrt{1-4\sin^2 t}}{2\cos t}.
\]
At $t=0$ we get
\[
k_2(0) = \frac{1}{2}.
\]
As $t\to\pi/6$, we have $\sin^2 t\to 1/4$, so $b'(t)\to0$ while
$\cos(\pi/6)=\sqrt{3}/2\neq0$, hence
\[
\lim_{t\to\pi/6}k_2(t)=0.
\]
Thus, $k_2$ is not constant, so the principal curvatures of $S$ are not
constant, even though $K\equiv 1$. 

\textbf{(ii)} For $w>0$, consider
\[
\gamma(t) = (a(t),b(t)) = \bigl(w\cosh(t/w),\,t\bigr),\qquad t\in\mathbb{R},
\]
and revolve it around the $z$–axis
\[
X(t,\theta) = \bigl(a(t)\cos\theta,\;a(t)\sin\theta,\;b(t)\bigr).
\]
Here, $a(t)=w\cosh(t/w)$, $b(t)=t$, so
\[
a'(t)=\sinh(t/w),\qquad a''(t)=\frac{1}{w}\cosh(t/w),\qquad
b'(t)=1,\quad b''(t)=0.
\]
Computing first derivatives
\[
X_t = (a'\cos\theta,\;a'\sin\theta,\;b'),\quad
X_\theta = (-a\sin\theta,\;a\cos\theta,\;0).
\]
So, the first fundamental form is 
\[
E = \langle X_t,X_t\rangle = a'^2 + b'^2
  = \sinh^2(t/w)+1 = \cosh^2(t/w),
\]
\[
F = \langle X_t,X_\theta\rangle = 0,\qquad
G = \langle X_\theta,X_\theta\rangle = a^2 = w^2\cosh^2(t/w).
\]
Then we have
\[
X_t\times X_\theta = (-ab'\cos\theta,\; -ab'\sin\theta,\; aa')
 =(-a\cos\theta,\;-a\sin\theta,\;aa'),
\]
so
\[
|X_t\times X_\theta| = a\sqrt{E},
\]
and a unit normal is
\[
N = \frac{1}{a\sqrt{E}}(X_t\times X_\theta)
  = \frac{1}{\sqrt{E}}(-b'\cos\theta,\; -b'\sin\theta,\;a')
  = \frac{1}{\cosh(t/w)}(-\cos\theta,\; -\sin\theta,\;\sinh(t/w)).
\]
Computing second derivatives
\[
X_{tt} = (a''\cos\theta,\;a''\sin\theta,\;b''),\quad
X_{t\theta} = (-a'\sin\theta,\;a'\cos\theta,\;0),\quad
X_{\theta\theta} = (-a\cos\theta,\;-a\sin\theta,\;0).
\]
Then the second fundamental form are
\[
e = \langle X_{tt},N\rangle
  = \frac{-a''b' + a'b''}{\sqrt{E}}
  = \frac{-a''}{\sqrt{E}},
\]
\[
f = \langle X_{t\theta},N\rangle = 0,
\]
\[
g = \langle X_{\theta\theta},N\rangle
  = \frac{ab'}{\sqrt{E}}
  = \frac{a}{\sqrt{E}}.
\]
Plug for $a$, $a''$, $E$:
\[
E = \cosh^2(t/w),\quad \sqrt{E} = \cosh(t/w),\quad
a = w\cosh(t/w),\quad a'' = \frac{1}{w}\cosh(t/w).
\]
Hence,
\[
e = -\frac{a''}{\sqrt{E}}
  = -\frac{\frac{1}{w}\cosh(t/w)}{\cosh(t/w)}
  = -\frac{1}{w},
\]
\[
g = \frac{a}{\sqrt{E}}
  = \frac{w\cosh(t/w)}{\cosh(t/w)}
  = w.
\]
So $e$ and $g$ are constants.
Since $F=f=0$, the shape operator matrix in the basis $\{X_t,X_\theta\}$ is
\[
S = \text{I}^{-1}\text{II}
  = \begin{pmatrix} E^{-1} & 0\\ 0 & G^{-1}\end{pmatrix}
    \begin{pmatrix} e & 0\\ 0 & g\end{pmatrix}
  = \begin{pmatrix}
      \dfrac{e}{E} & 0\\[4pt]
      0 & \dfrac{g}{G}
    \end{pmatrix}.
\]
Thus, the principal curvatures are
\[
k_1 = \frac{e}{E}
         = \frac{-1/w}{\cosh^2(t/w)}
         = -\frac{1}{w\cosh^2(t/w)},
\]
\[
k_2 = \frac{g}{G}
         = \frac{w}{w^2\cosh^2(t/w)}
         = \frac{1}{w\cosh^2(t/w)}.
\]
Therefore,
\[
k_1 + k_2 = 0,
\]
so the mean curvature
\[
H = \frac{k_1 + k_2}{2} = 0.
\]
This implies $M_w$ is a minimal surface for every $w>0$.
\end{proof}


\section*{Problem 4}
\begin{proof}
\textbf{(i)} Fix a point $p\in M$ and write $V=T_pM$ with inner product 
induced by $g$.
For $w,x,y,z\in V$, define
\[
B(w,x,y,z)
:= \langle w,y\rangle\langle x,z\rangle
   -\langle w,z\rangle\langle x,y\rangle .
\]
For fixed $y,z$, this is alternating in $(w,x)$, and for fixed $w,x$ it is
alternating in $(y,z)$.  
By property of the exterior product in Lee's review, there exists a unique bilinear map
\[
\langle\cdot,\cdot\rangle_{\Lambda^2} : \Lambda^2 V \times \Lambda^2 V \to \mathbb{R}
\]
such that
\[
\langle w\wedge x,\; y\wedge z\rangle_{\Lambda^2}
= B(w,x,y,z)
= \langle w,y\rangle\langle x,z\rangle
 -\langle w,z\rangle\langle x,y\rangle
\]
for all $w,x,y,z\in V$.

We first check that $\langle\cdot,\cdot\rangle_{\Lambda^2}$ is an inner
product.  
Let $\{e_1,\dots,e_n\}$ be an orthonormal basis of $V$. Then
\[
\langle e_i\wedge e_j,\; e_k\wedge e_\ell\rangle_{\Lambda^2}
= \delta_{ik}\delta_{j\ell} - \delta_{i\ell}\delta_{jk}.
\]
In particular, for $i<j$ and $k<\ell$, this is $1$ if $(i,j)=(k,\ell)$ and $0$
otherwise.  
Thus, the family
\[
\{e_i\wedge e_j : 1\le i<j\le n\}
\]
is an orthonormal basis of $\Lambda^2V$, so
$\langle\cdot,\cdot\rangle_{\Lambda^2}$ is symmetric and positive definite.
Hence, it is an inner product on $\Lambda^2V$.
Then we compute the associated norm on decomposable $2$-vectors
\[
|w\wedge x|^2
= \langle w\wedge x,\,w\wedge x\rangle_{\Lambda^2}
= \langle w,w\rangle\langle x,x\rangle
 -\langle w,x\rangle^2
= |w|^2|x|^2 - \langle w,x\rangle^2.
\]
This shows the existence.

For uniqueness, suppose $(\cdot,\cdot)'$ is any other fiber metric on
$\Lambda^2V$ whose associated norm satisfies
\[
|w\wedge x|'^2 = |w|^2|x|^2 - \langle w,x\rangle^2
\quad\text{for all }w,x\in V.
\]
Let $\{e_1,\dots,e_n\}$ be an orthonormal basis of $V$.
For $i<j$, we have
\[
|e_i\wedge e_j|'^2
= |e_i|^2|e_j|^2 - \langle e_i,e_j\rangle^2
= 1,
\]
so each $e_i\wedge e_j$ has norm $1$ with respect to $(\cdot,\cdot)'$.
To see that they are orthogonal, fix distinct $i,k$ and some $j$.
Take $w=e_i+e_k$ and $x=e_j$. Then
\[
w\wedge x = (e_i+e_k)\wedge e_j
          = e_i\wedge e_j + e_k\wedge e_j.
\]
On one hand, by the given norm formula and orthonormality of $e_i,e_j,e_k$,
\[
|w\wedge x|'^2
= |w|^2|x|^2 - \langle w,x\rangle^2
= 2\cdot1 - 0 = 2.
\]
On the other hand, expanding with the (unknown) inner product,
\[
|w\wedge x|'^2
= |e_i\wedge e_j + e_k\wedge e_j|'^2
= |e_i\wedge e_j|'^2 + |e_k\wedge e_j|'^2
  + 2(e_i\wedge e_j,\,e_k\wedge e_j)'.
\]
Since each norm is $1$, this is
\[
2 = 1+1 + 2(e_i\wedge e_j,\,e_k\wedge e_j)'
\quad\Longrightarrow\quad
(e_i\wedge e_j,\,e_k\wedge e_j)' = 0.
\]
Similar choices of $w,x$ show that
$(e_i\wedge e_j,\,e_k\wedge e_\ell)' = 0$ whenever $(i,j)\ne(k,\ell)$ with
$i<j$, $k<\ell$.  
Thus, $\{e_i\wedge e_j\}_{i<j}$ is an orthonormal basis for $(\cdot,\cdot)'$ as
well.
But on a finite-dimensional vector space, an inner product is uniquely
determined by declaring a basis to be orthonormal. Hence, $(\cdot,\cdot)'$
coincides with $\langle\cdot,\cdot\rangle_{\Lambda^2}$.  
Therefore, the fiber metric constructed above is the unique one whose associated
norm satisfies
\[
|w\wedge x|^2 = |w|^2|x|^2 - \langle w,x\rangle^2.
\]

\textbf{(ii)} Fix $p\in M$ and write $V=T_pM$.
Denote by $\mathrm{Rm}_p$ the curvature tensor at $p$:
\[
\mathrm{Rm}_p(w,x,y,z) = g_p(R(w,x)y,z).
\]
Then it is multilinear and alternating in $(w,x)$ and in $(y,z)$.
Define first a bilinear form
\[
B : \Lambda^2V \times \Lambda^2V \to \mathbb{R}
\]
by
\[
B(w\wedge x,\; y\wedge z) := -\mathrm{Rm}_p(w,x,y,z),
\]
and extend by bilinearity.  
This is well-defined by the universal property of $\Lambda^2V$, because
$\mathrm{Rm}_p$ is alternating in both $(w,x)$ and $(y,z)$.

Now $(\Lambda^2V,\langle\cdot,\cdot\rangle_{\Lambda^2})$ is a finite-dimensional
inner product space. By linear algebra, any bilinear form $B$ on such a space is represented by a
unique linear operator $R_p : \Lambda^2V\to\Lambda^2V$ via
\[
B(\alpha,\beta) = \langle R_p(\alpha),\beta\rangle_{\Lambda^2}
\qquad\text{for all }\alpha,\beta\in\Lambda^2V.
\]
Thus, there exists a unique linear map
\[
R_p : \Lambda^2 T_pM \to \Lambda^2 T_pM
\]
such that
\[
\langle R_p(w\wedge x),\, y\wedge z\rangle_{\Lambda^2}
= -\mathrm{Rm}_p(w,x,y,z)
\]
for all $w,x,y,z\in T_pM$.

These maps $R_p$ vary smoothly with $p$. In a local orthonormal frame
$\{E_i\}$ of $TM$, the frame $\{E_i\wedge E_j\}$ of $\Lambda^2(TM)$ is
orthonormal, and we can write
\[
R_p(E_i\wedge E_j)
= \sum_{k<\ell} \bigl(-R_{ij k\ell}(p)\bigr)\,E_k\wedge E_\ell,
\]
where $R_{ijk\ell}$ are the components of the curvature tensor; these
coefficients are smooth functions.  
Thus, $R$ is a smooth bundle endomorphism
\[
\mathcal{R} : \Lambda^2(TM) \to \Lambda^2(TM),
\]
called the curvature operator of $g$, and it satisfies
\[
\langle \mathcal{R}(w\wedge x),\,y\wedge z\rangle
= -\mathrm{Rm}(w,x,y,z)
\]
for all tangent vectors $w,x,y,z$.
Also, the uniqueness of $\mathcal{R}$ follows from the fiberwise uniqueness of $R_p$.
\end{proof}

\section*{Problem 5}
\begin{proof}
Write
\[
v(t) := \gamma'(t),\qquad a(t):=D_t\gamma'(t)
\]
for its velocity and covariant acceleration.  Set
\[
T(t) := \frac{v(t)}{|v(t)|},
\]
the unit tangent field along $\gamma$.
Choose an arc-length parameter $s$ for $\gamma$, so that
\[
\sigma(s) := \gamma(t(s)), \qquad |\sigma'(s)| \equiv 1.
\]
By def, the geodesic curvature of $\gamma$ at $t$ is 
\[
k(t) := \bigl|D_s T\bigr|_{s=s(t)} ,
\]
where $T=\sigma'$ regarded as a unit tangent field along the
unit-speed reparametrization.
We express $D_sT$ in terms of $t$.  Since
\[
\frac{ds}{dt} = |v(t)|, \qquad \frac{d}{ds} = \frac{1}{|v|}\,D_t,
\]
we obtain
\[
D_s T = \frac{1}{|v|}\,D_t T.
\]
Now compute $D_tT$.  Using $T = v/|v|$ and the product rule,
\[
D_t T
= D_t\Bigl(\frac{v}{|v|}\Bigr)
= \frac{1}{|v|}a - \frac{1}{|v|^3}\,\langle a,v\rangle\,v.
\]
Define the component of $a$ orthogonal to $v$ by
\[
a_\perp := a - \frac{\langle a,v\rangle}{|v|^2}v.
\]
Then
\[
D_t T = \frac{1}{|v|}a_\perp,
\qquad
D_s T = \frac{1}{|v|^2}a_\perp.
\]
Hence
\[
k(t) = |D_sT| = \frac{|a_\perp|}{|v|^2}.
\]
By def of $a_\perp$,
\[
|a_\perp|^2
= |a|^2 - \frac{\langle a,v\rangle^2}{|v|^2}.
\]
On the other hand, by Problem~4,
\[
|v\wedge a|^2
= |v|^2|a|^2 - \langle v,a\rangle^2.
\]
Combining these,
\[
|a_\perp|^2
= \frac{|v\wedge a|^2}{|v|^2}.
\]
Therefore,
\[
k(t)^2
= \frac{|a_\perp|^2}{|v|^4}
= \frac{|v\wedge a|^2}{|v|^6},
\]
and taking square roots gives
\[
k(t) = \frac{|\,\gamma'(t)\wedge D_t\gamma'(t)\,|}{|\gamma'(t)|^3}.
\]
Here the norm in the numerator is the one induced on $\Lambda^2(TM)$ as in
Problem~4.
Now assume $M=\mathbb{R}^3$ with the Euclidean metric.  There is a natural
identification between $2$--vectors and vectors via the cross product.
For $u,v\in\mathbb{R}^3$,
$
u\times v
\text{ corresponds to }
u\wedge v\in\Lambda^2\mathbb{R}^3,
$
and this identification is an isometry, so
\[
|u\times v| = |u\wedge v|.
\]
Thus, for $\gamma:I\to\mathbb{R}^3$,
\[
k(t)
= \frac{|\,\gamma'(t)\wedge \gamma''(t)\,|}{|\gamma'(t)|^3}
= \frac{|\,\gamma'(t)\times \gamma''(t)\,|}{|\gamma'(t)|^3}.
\]

\end{proof}

\section*{Problem 6}
\begin{proof}
    Fix $p\in M$ and write $V=T_pM\subset\mathbb{R}^{n+1}$.
For any $X\in V$, consider $N$ as a vector field along $M$ in $\mathbb{R}^{n+1}$.
Then
\[
d\nu_p(X) = D_X N \in T_{\nu(p)}S^n\subset\mathbb{R}^{n+1},
\]
where $D$ is the ordinary derivative in $\mathbb{R}^{n+1}$.

Because $|N|^2\equiv1$, we have
\[
0 = X\bigl(\langle N,N\rangle\bigr)
  = 2\langle D_XN, N\rangle,
\]
so $D_XN\perp N$.  Since both $T_pM$ and $T_{\nu(p)}S^n$ are the hyperplane
$\{v\in\mathbb{R}^{n+1} : \langle v,N_p\rangle=0\}$, we have
$D_XN\in T_pM$ and $D_XN\in T_{\nu(p)}S^n$ simultaneously.

By definition of the shape operator $S$ of $M$ (Lee, §8.2),
\[
S_p(X) = -D_XN \in T_pM.
\]
Thus, viewing $T_pM$ and $T_{\nu(p)}S^n$ as the same subspace of
$\mathbb{R}^{n+1}$,
\[
\,d\nu_p = -S_p : T_pM\to T_{\nu(p)}S^n.\,
\]

%------------------------------
% Step 2: Pullback of volume form
%------------------------------
Both $T_pM$ and $T_{\nu(p)}S^n$ are $n$–dimensional oriented inner product
spaces, with volume forms $(dV_g)_p$ and $(dV_{g_0})_{\nu(p)}$.
Let $e_1,\dots,e_n$ be an oriented orthonormal basis of $T_pM$.
Since $N_p$ is the outward unit normal for both $M$ and $S^n$, we may regard
$e_1,\dots,e_n$ also as an oriented orthonormal basis of $T_{\nu(p)}S^n$
(the condition that $(N_p,e_1,\dots,e_n)$ is an oriented orthonormal basis of
$\mathbb{R}^{n+1}$ is the same in both cases).

By definition of the pullback of a top-degree form,
\[
(\nu^*dV_{g_0})_p(e_1,\dots,e_n)
 = (dV_{g_0})_{\nu(p)}\!\bigl(d\nu_p(e_1),\dots,d\nu_p(e_n)\bigr).
\]
In an oriented orthonormal basis, a volume form is just the determinant:
\[
(dV_{g_0})_{\nu(p)}\bigl(d\nu_p(e_1),\dots,d\nu_p(e_n)\bigr)
 = \det(d\nu_p)\,(dV_{g_0})_{\nu(p)}(e_1,\dots,e_n).
\]
But $(dV_{g_0})_{\nu(p)}(e_1,\dots,e_n)=1$ and
$(dV_g)_p(e_1,\dots,e_n)=1$, so
\[
(\nu^*dV_{g_0})_p(e_1,\dots,e_n)
 = \det(d\nu_p)
 = \det(-S_p).
\]

If $\kappa_1,\dots,\kappa_n$ are the principal curvatures of $M$ at $p$, then
$S_p$ is diagonalizable in an orthonormal basis with eigenvalues
$\kappa_1,\dots,\kappa_n$, so
\[
\det(S_p) = \kappa_1\cdots\kappa_n =: K(p)
\]
is the Gaussian curvature of $M$ at $p$ (product of principal curvatures).
Hence
\[
\det(d\nu_p) = \det(-S_p) = (-1)^n\det(S_p) = (-1)^n K(p).
\]

Therefore,
\[
(\nu^*dV_{g_0})_p(e_1,\dots,e_n)
 = (-1)^n K(p)
 = \bigl((-1)^n K\,dV_g\bigr)_p(e_1,\dots,e_n).
\]
Since this holds for all oriented orthonormal bases of $T_pM$, we conclude
\[
\;\nu^* dV_{g_0} = (-1)^n K\, dV_g.\;
\]

\end{proof}
\end{document}
